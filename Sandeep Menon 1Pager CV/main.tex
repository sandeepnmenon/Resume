%-------------------------
% Resume in Latex
% Author : Jake Gutierrez
% Based off of: https://github.com/sb2nov/resume
% License : MIT
%------------------------

\documentclass[12pt]{article}

\usepackage{latexsym}
\usepackage[empty]{fullpage}
\usepackage{titlesec}
\usepackage{marvosym}
\usepackage[usenames,dvipsnames]{color}
\usepackage{verbatim}
\usepackage{enumitem}
\usepackage[hidelinks]{hyperref}
\usepackage{fancyhdr}
\usepackage[english]{babel}
\usepackage{tabularx}
\usepackage{fontawesome5}
\usepackage{multicol}
\setlength{\multicolsep}{-3.0pt}
\setlength{\columnsep}{-1pt}
\input{glyphtounicode}


%----------FONT OPTIONS----------
%sans-serif
% \usepackage[sfdefault]{FiraSans}
% \usepackage[sfdefault]{helvet}
\usepackage[sfdefault]{roboto}
% \usepackage[sfdefault]{noto-sans}
% \usepackage[default]{segoe}
% \usepackage[default]{sourcesanspro}

% serif
% \usepackage{CormorantGaramond}
% \usepackage{charter}


\pagestyle{fancy}
\fancyhf{} % clear all header and footer fields
\fancyfoot{}
\renewcommand{\headrulewidth}{0pt}
\renewcommand{\footrulewidth}{0pt}

% Adjust margins
% \addtolength{\oddsidemargin}{-0.6in}
% \addtolength{\evensidemargin}{-0.6in}
% \addtolength{\textwidth}{1.19in}
% \addtolength{\topmargin}{-.5in}
% % \addtolength{\bottmargin}{-.5in}
% \addtolength{\textheight}{1.4in}

\usepackage[hmargin=1cm, vmargin=1cm]{geometry}
\usepackage{xcolor}
\usepackage[colorlinks = true,
            linkcolor = blue,
            urlcolor  = blue,
            citecolor = blue,
            anchorcolor = blue]{hyperref}
\newcommand{\MYhref}[3][blue]{\href{#2}{\color{#1}{#3}}}%

\urlstyle{same}

\raggedbottom
\raggedright
\setlength{\tabcolsep}{0in}

\definecolor{title-rule-color}{HTML}{808080}
% Sections formatting
\titleformat{\section}{
  \vspace{-4pt}\scshape\raggedright\large\bfseries
}{}{em}{}[\color{title-rule-color}\titlerule \vspace{-5pt}]

% Ensure that generate pdf is machine readable/ATS parsable
\pdfgentounicode=1

%-------------------------
% Custom commands
\newcommand{\resumeItem}[1]{
  \item\small{
    {#1 \vspace{-2pt}}
  }
}

\newcommand{\classesList}[4]{
    \item\small{
        {#1 #2 #3 #4 \vspace{-2pt}}
  }
}
\definecolor{italic-color}{HTML}{555555}

\newcommand{\resumeEducation}[4]{
  \vspace{-2pt}\item
    \begin{tabular*}{1.0\textwidth}[t]{l@{\extracolsep{\fill}}r}
      \textbf{#1} & \textbf{\small #2} \\
      \text{\small#3} & \text{\color{italic-color}\small #4} \\
      % \text{\small #5}
    \end{tabular*}\vspace{-7pt}
}

\newcommand{\resumeMooc}[3]{
  \vspace{-2pt}\item
    \begin{tabular*}{1.0\textwidth}[t]{l@{\extracolsep{\fill}}r}
      \textbf{#1} & \textbf{\small #2} \\
    \end{tabular*}
    \text{\small#3}
}

\newcommand{\resumeSubheading}[2]{
  \vspace{-2pt}\item
    \begin{tabular*}{1.0\textwidth}[t]{l@{\extracolsep{\fill}}r}
      \textbf{#1} & \textbf{\small #2} \\
      % \text{\small #3} & \text{\color{italic-color}\small #4} \\
    \end{tabular*}\vspace{-7pt}
}

\newcommand{\resumeSubSubheading}[2]{
    \item
    \begin{tabular*}{0.97\textwidth}{l@{\extracolsep{\fill}}r}
      \textit{\small#1} & \textit{\small #2} \\
    \end{tabular*}\vspace{-7pt}
}

\newcommand{\resumeProjectHeading}[2]{
    \item
    \begin{tabular*}{1.001\textwidth}{l@{\extracolsep{\fill}}r}
      \small#1 & \textbf{\small #2}\\
    \end{tabular*}\vspace{-7pt}
}

\newcommand{\resumeSubItem}[1]{\resumeItem{#1}}

\renewcommand\labelitemi{$\vcenter{\hbox{\tiny$\bullet$}}$}
\renewcommand\labelitemii{$\vcenter{\hbox{\tiny$\bullet$}}$}

\newcommand{\resumeSubHeadingListStart}{\begin{itemize}[leftmargin=0.0in, label={}]\itemsep1pt}
\newcommand{\resumeSubHeadingListEnd}{\end{itemize}}
\newcommand{\resumeItemListStart}{\begin{itemize}\itemsep1pt}
\newcommand{\resumeItemListEnd}{\end{itemize}}

\usepackage{bibentry} 
\begin{filecontents}{publication.bib}
@inproceedings{menon2020novel,
  title={A novel deep learning approach for the removal of speckle noise from optical coherence tomography images using gated convolution--deconvolution structure},
  author={\textbf{Sandeep N Menon} and Reddy, VB Vineeth and Yeshwanth, A and Anoop, BN and Rajan, Jeny},
  booktitle={Proceedings of 3rd International Conference on Computer Vision and Image Processing},
  pages={115--126},
  year={2020},
  organization={Springer, Singapore}
}
@INPROCEEDINGS{oct2019,
  title={A novel Deep Learning Approach for the removal ofspeckle noise from Optical Coherence Tomography Images using Gated Convolution Deconvolution Structure},
  author={Sandeep N Menon, V B Vineeth Reddy, Yeshwanth A, Anoop B N and Dr Jeny Rajan},
  year={2019},
  publisher={Springer},

}
\end{filecontents}
\usepackage{lastpage}
\usepackage{fancyhdr}
\usepackage{ifthen}

% \cfoot{\thepage\ of \pageref{LastPage}}
\fancyfoot[L]{\ifnum\thepage=2 \small \today\ \currenttime \fi}
    % \fancyfoot[L]{\today\ \currenttime}
% \fancyhead[R]{\today}
\fancyfootoffset[R]{10pt}
\fancyfoot[R]{\small \thepage\ of \pageref{LastPage}}

\newcommand{\lastupdated}{\begin{textblock}
\fontsize{8pt}{10pt}\selectfont 
\today
\end{textblock}}

%-------------------------------------------
%%%%%%  RESUME STARTS HERE  %%%%%%%%%%%%%%%%%%%%%%%%%%%%

% ATS friendly pdf as per https://www.reddit.com/r/LaTeX/comments/3nr2vn/how_can_i_ensure_that_my_latex_resume_is_readable/
\input{glyphtounicode}
\pdfgentounicode=1

\begin{document}
\bibliographystyle{plain}
\nobibliography{publication}
%----------HEADING----------
% \begin{tabular*}{\textwidth}{l@{\extracolsep{\fill}}r}
%   \textbf{\href{http://sourabhbajaj.com/}{\Large Sourabh Bajaj}} & Email : \href{mailto:sourabh@sourabhbajaj.com}{sourabh@sourabhbajaj.com}\\
%   \href{http://sourabhbajaj.com/}{http://www.sourabhbajaj.com} & Mobile : +1-123-456-7890 \\
% \end{tabular*}

\begin{center}
    {\Huge \scshape Sandeep N Menon} \\ \vspace{1pt}
    % Flat No 6B, Rubber Residency, Kanjikuzhy Kottayam, Kerala, India 686004 \\ \vspace{1pt}
    \small \raisebox{-0.1\height}\faPhone\ +1 347-951-5417 ~ \href{mailto:menonsandu@gmail.com}{\raisebox{-0.2\height}\faEnvelope\  {menonsandu@gmail.com}} ~ 
    \href{https://linkedin.com/in/sandeep-n-menon}{\raisebox{-0.2\height}\faLinkedin\ {linkedin.com/in/sandeep-n-menon}}  \\
    \href{https://github.com/sandeepnmenon}{\raisebox{-0.2\height}\faGithub\ {github.com/sandeepnmenon}} \href{https://sandeepnmenon.github.io/}{\raisebox{-0.2\height}\faHome \ sandeepnmenon.github.io}
    \vspace{-10pt}
\end{center}


%-----------Research Interests-----------
% \section{Research Interests}
% Deep Learning, Computer Vision, Adversarial Networks, Generative Models

%-----------EDUCATION-----------
\section{Education}
  \resumeSubHeadingListStart
    \resumeEducation{New York University (NYU) Courant Institute of Mathematical Sciences}{2022 -- Present}{Masters in Computer Science}{New York, USA}
    \resumeEducation
      {National Institute of Technology Karnataka, Surathkal, India (NITK)}{2014 -- 2018}
      {Bachelor of Technology in Computer Science - CGPA: 8.83/10}{Karnataka, India}
      \resumeItemListStart
                \resumeItem{President of Web Enthusiasts Club NITK. Organized mock technical interviews, CTFs, Linux installation drives.}
                \vspace{-16pt}
                \resumeItem{Core executive member at IEEE NITK Student Chapter. Conducted hackathons and programming contests.}
                % \resumeItem{Writing research paper commentary at \hyperlink{https://oneortworesearchdigest.blogspot.com}{\textit{https://oneortworesearchdigest.blogspot.com}} focusing on new neural network architectures and applications since 2020.}
        \resumeItemListEnd
    \vspace{-10pt}
  \resumeSubHeadingListEnd

% %------RELEVANT COURSEWORK-------
% \section{Relevant Coursework}
%     %\resumeSubHeadingListStart
%         \begin{multicols}{4}
%             \begin{itemize}[itemsep=-5pt, parsep=3pt]
%                 \item\small Computer Vision
%                 \item Artificial Intelligence
%                 \item Linear Algebra
%                 \item Probability Theory
%                 \item Advanced Data Structures
%                 \item Graph Theory
%                 \item Digital Image Processing
%                 \item Computer Graphics
%             \end{itemize}
%         \end{multicols}
%         \vspace*{2.0\multicolsep}
%     %\resumeSubHeadingListEnd


%-----------EXPERIENCE-----------
\section{Industry Experience}
  \resumeSubHeadingListStart
    \resumeSubheading
      {Deep Learning Research Engineer $|$ \href{https://www.deepen.ai/}{Deepen AI} $|$
 Hyderabad, India}{September 2020 -- July 2022}
      % {Deepen AI}{}
      \resumeItemListStart
        \resumeItem{Developed a 3D PointNet model that performs temporal smoothing of segmentation predictions over point cloud sequences, improving mean Intersection over Union (mIoU) by 20\%.}
        \resumeItem{Built a Sparse Point-Voxel CNN model for semantic segmentation of 3D point cloud sequences. Improved data annotation speed by 30\% and achieved mIoU score of 76\%.}
        \resumeItem{Implemented object-aware anchor-free tracking for 2D visual object tracking and VPGNet model for lane segmentation and classification.}
        \resumeItem{Developed algorithm for targetless Camera-IMU and stereo camera calibration.}
        % \resumeItem{Implemented algorithm for targetless stereo camera calibration.}
        % \resumeItem{Created an on-demand GPU Virtual Machine allocation system using Azure. Enabled automatic allocation and de-allocation of expensive GPU machines, thereby saving up to 1000 USD per month.}
        % \resumeItem{Worked with Potree, an open-source WebGL based point cloud renderer for large point clouds, thereby developing a system that could render point clouds with more than 150 million points in a web browser.}
      \resumeItemListEnd

    \resumeSubheading
      {Software Development Engineer II $|$ \href{https://www.microsoft.com/en-in}{Microsoft} $|$ Hyderabad, India}{June 2018 -- September 2020}
      % {Microsoft}{}
      \resumeItemListStart
        \resumeItem{Introduced a new Machine Learning method to identify similar won deals in CRM context for Relationship Analytics in Dynamics 365, received a patent award on the same.}
        % \resumeItem{Designed and implemented Machine Learning infrastructure to run, store and display AI model predictions using Azure Batch, MySQL, and .NET Core. } % metrics and reword for non-MS readers.
        \resumeItem{Developed a GDPR query handling service for the email insights infrastructure that handles up to 1 million daily service requests.}
        \resumeItem{Shipped Dynamics 365 sales insights connector in Microsoft Flows that manages more than 9 million monthly service requests. }
    \resumeItemListEnd
    
  \resumeSubHeadingListEnd
\vspace{-16pt}


% Reduce spacing
% give dates
%-----------PROJECTS-----------
\section{Selected Publications and Projects}
    \vspace{-5pt}
    \resumeSubHeadingListStart
      \resumeProjectHeading
          {\textbf{\href{https://link.springer.com/chapter/10.1007/978-981-32-9291-8_10}{Removing noise from Optical Coherence Tomography (OCT) Images} [published]}}{August 2017 - May 2018}
        %   \vspace{-15pt}
          \resumeItemListStart
            \resumeItem{\bibentry{menon2020novel}}
            % \resumeItem{Created a deep encoder-decoder neural network with gated skip connections that identifies and removes noise from medical images. Applied this method on OCT images to eliminate the inherent speckle noise.}
            % \resumeItem{Achieved Structural Similarity Index (SSIM) value of 96.7\% for low noise images and 91.2\% for high noise images, surpassing the state-of-the-art results at the time of publishing.}
          \resumeItemListEnd
          \vspace{-20pt}

        \resumeProjectHeading{\textbf{Point Cloud Oversegmentation using Superpoint Graphs} $|$ \emph{PyTorch, Boost}}{May - June 2021}
        \resumeItemListStart
            \resumeItem{Adapted Superpoint Graph implementation to Argoverse point cloud dataset to achieve oversegmentation results of overall accuracy of 96\% and Boundary Recall of 92\%.}
        \resumeItemListEnd
          \vspace{-20pt}
        % \resumeProjectHeading{\textbf{Point Cloud Segmentation using projected 2D segmentation} $|$ \emph{MMSegmentation}}{April - May 2021}
        % \resumeItemListStart
        %     \resumeItem{Created point cloud semantic segmentation pipeline by projecting image segmentation of camera output using HRNet, DeepLabV3 and PointRend models.}
        %     \resumeItem{Implemented projection of 3D space onto the 2D segmentation results of images from every camera mount to obtain the estimated class value of all 3D points; inspired from PointPainting paper.}
        % \resumeItemListEnd
        % \vspace{-15pt}
        \resumeProjectHeading{\textbf{Online calibration of Surround-view Camera system} $|$ \emph{OpenCV, Sophus, Boost}}{April - May 2021}
        \resumeItemListStart
            \resumeItem{Online calibration of the four surround-view camera system by minimizing photometric lose in the overlapping regions of the bird-eye view. Calibration possible with just one snapshot from the four cameras.}
        \resumeItemListEnd
        \vspace{-20pt}
        \resumeProjectHeading
        {\textbf{\href{https://github.com/mit-han-lab/torchsparse/pull/28}{Asymmetric 3D Convolutions in Torchsparse}} $|$ \emph{PyTorch}}{February 2021}
        \resumeItemListStart
            \resumeItem{Contributed Asymmetric 3D Convolutions implementation for the open source repository \hyperlink{https://github.com/mit-han-lab/torchsparse}{TorchSparse}}
        \resumeItemListEnd
          \vspace{-20pt}
      \resumeProjectHeading
          {\textbf{\href{https://github.com/durovo/edgehack}{Virtual Gym Trainer}} $|$ \emph{PyTorch, Azure, OpenCV, Pose Estimation} $|$ \href{https://www.youtube.com/watch?v=VLFEikGUWUI}{Demo link}}{May - June 2019}
          \resumeItemListStart
            \resumeItem{Platform for guiding users through trainer-specified exercises using automatic audio and visual cues.}
            % \resumeItem{Provided real-time human pose estimation using PoseNet network and geometric estimations to measure correctness of the posture.}
            % \resumeItem{Awarded 1$^{st}$ position in Microsoft Intelligent Edge Hackathon 2019.}
          \resumeItemListEnd 

    \resumeSubHeadingListEnd
\vspace{-15pt}

%
%-----------PROGRAMMING SKILLS-----------
\section{Technical Skills}
 \begin{itemize}[leftmargin=0.0in, label={}]
    \small{\item{
    \textbf{Strengths}{: Deep Learning (PyTorch, TensorFlow, Keras), Cloud Computing (Azure, Google Cloud Platform)} \\
     \textbf{Languages/Platforms}{: C++, C\#, Python, Go, React, Docker, MongoDB, RocksDB, MySQL, Cosmos DB} \\
     % \textbf{Cloud}{: Azure, Google Cloud Platform} \\
    }}
 \end{itemize}
   \vspace{-16pt}

%------RELEVANT COURSEWORK-------
% \section{Certifications}
%   \resumeSubHeadingListStart
%     \resumeMooc
%       {\href{https://graduation.udacity.com/confirm/HV7HCJHC}{Computer Vision Nanodegree, Udacity}}{June 2020 -- August 2020}
%       {Image Captioning using CNN-RNN, Landmark Detection and Tracking using 2D Graph SLAM}
%     \resumeMooc{\href{https://www.coursera.org/account/accomplishments/specialization/CT6ZUGU494YJ}{Deeplearning Specialization}, deeplearning.ai}{December 2020 - March 2020}
%     {Neural Networks and Deep learning, Structuring Machine Learning Projects, Convolutional Neural Networks, Sequence Models}
%     \resumeMooc{\href{https://courses.edx.org/certificates/97611db8697a4dc1a46e3f428e28ae25}{Statistical Learning}, StanfordOnline: STATSX0001 }{March 2020 -- April 2020}{Boosting trees,
% Discriminant Analysis, Splines,
% Support vector machines}
%   \resumeSubHeadingListEnd

%  \vspace{-16pt}


%-----------INVOLVEMENT---------------
% \section{Leadership / Extracurricular}
%     % \resumeSubHeadingListStart
%         % \resumeSubheading{Fraternity}{Spring 2020 -- Present}{President}{University Name}
%             \resumeItemListStart
%                 \resumeItem{President of Web Enthusiasts Club NITK. Organized mock technical interviews, CTFs, Linux installation drives.}
%                 \vspace{-16pt}
%                 \resumeItem{Core executive member at IEEE NITK Student Chapter. Conducted hackathons and programming contests.}
%                 % \resumeItem{Writing research paper commentary at \hyperlink{https://oneortworesearchdigest.blogspot.com}{\textit{https://oneortworesearchdigest.blogspot.com}} focusing on new neural network architectures and applications since 2020.}
%             \resumeItemListEnd
        
    % \resumeSubHeadingListEnd


\end{document}
